\documentclass[10pt,a4paper,sans]{moderncv}

% ModernCV themes
\moderncvstyle{classic} % Style options: 'casual', 'classic', 'oldstyle' and 'banking'
\moderncvcolor{blue}    % Color options: 'blue' (default), 'orange', 'green', 'red', 'purple', 'grey' and 'black'

% Character encoding
%\usepackage[utf8]{inputenc} % This is required for xelatex or lualatex
\usepackage[T1]{fontenc}
\usepackage{fontspec} % This is required for xelatex or lualatex

% Adjust the page margins
\usepackage[scale=0.95]{geometry}
\renewcommand{\baselinestretch}{0.90} % Reduces line spacing
\setlength{\hintscolumnwidth}{2.5cm} % Adjust the width as needed


% Personal data
\name{Marius}{Lhôte}
\title{Curriculum Vitae}
\address{Lausanne, Switzerland}
\email{marius.lhote@epfl.ch}
\social[linkedin]{marius-lhote} % optional
\social[github]{Autek} % optional
\photo[80pt][0.4pt]{picture1.jpg} % optional, adjust the dimensions if needed

%----------------------------------
% Content
%----------------------------------
\begin{document}

\makecvtitle

\section{Education}

\cventry{2017--2021}{Maturité Gymnasiale - OS Biologie et Chimie}{Collège Claparède}{Genève, Switzerland}{}{
	% Description
	\begin{itemize}
		\item Relevant coursework: Chemistry, Biology.
	\end{itemize}
}
\cventry{2021--Present}{Bachelor in Communication Systems}{EPFL}{Lausanne, Switzerland}{}{
  % Description
  \textit{Expected Graduation: 2025}
  \begin{itemize}
	\item First-year grade average: 5.43/6
	\item Relevant coursework: Introduction to Machine Learning, Numerical Methods for Visual Computing and Machine Learning, Signal Processing, Analysis(I, II, III, IV), Probability and Statistics, Algorithm, electromagnetism.
  \end{itemize}
}

\section{Academic Projects}
\cventry{}{Machine Learning Projects}{Introduction to Machine Learning}{}{}{
	\begin{itemize}
		\item Implemented in a team of 3 a linear regression, a logistic regression, a knn classifier from scratch in Python using NumPy.
		\item located centers and classified the Stanford Dogs dataset with an accuracy of 87\%.
		\item Implemented in a team of 3 a MLP, a CNN and a ViT in Python using PyTorch.
		\item classified the Fashion MNIST dataset with an accuracy of 90\%.
    \end{itemize}
}
\cventry{}{Copy of flightradar24}{Pratique de la Programmation Orientée Objet}{}{}{
	\begin{itemize}
		\item Implemented a copy of flightradar24 in Java using JavaFX in a team of 2.
		\item learned about the Observer pattern, the MVC pattern, and how to use JavaFX.
		\item learned how to use the OpenStreetMap API to display a map.
		\item optimized the code to process planes data faster than real-time.
	\end{itemize}
}

\section{Personal Projects}
\cventry{}{\href{https://github.com/Autek/go_of_life}{Game of Life in GO}}{}{}{}{
	\begin{itemize}
		\item Implemented Conway's Game of Life in GO using the Fyne library.
		\item Implemented an infinite grid that can be scrolled and zoomed.
		\item learned to write code and think about the architecture of the code without guidance.
	\end{itemize}
}

\section{Associative Experience}
\cventry{November 2023-- \linebreak Present}{Sponsorship Manager}{EPFL Nuit de la Magistrale}{}{}{
  \begin{itemize}
    \item Secured sponsorships worth over CHF 20'000 increasing the event's budget by 10\%.
	\item wrote a mailer in Python to automate the sending of personalized emails to sponsors. 
	\item sent 1000 personalized emails to potential sponsors and received a 10\% response rate.
  \end{itemize}
}

\section{Teaching and Peer Support Experience}
\cventry{September 2023-- \linebreak December 2023}{Mentor}{EPFL}{}{}{
  \begin{itemize}
	\item was responsible for 13 first-year students.
    \item Provided guidance and support to help students adapt to university academic life.
    \item Organized mentoring sessions to help students with their coursework.
    \item Organized and gave a course on how to use Git to 100 students.
  \end{itemize}
}
\cventry{February 2024-- \linebreak June 2024}{Teaching Assistant, MAN - Informatique et Calcul Scientifique}{EPFL}{}{}{
  \begin{itemize}
	\item I was the only teaching assistant in a room of 10 students.
    \item Helped students with their homework during every exercise sessions,
	\item Responded to 20\% of the students' questions on the course forum.
  \end{itemize}
}

\section{Skills}
\cvitem{Programming Languages}{Python, Scala, Java, GO, C}
\cvitem{Frameworks}{PyTorch, NumPy, Matplotlib, Scikit-learn}
\cvitem{Tools}{Git, Jupyter Notebook, Google Colab, yaml, LaTeX}

\section{Languages}
\cvitemwithcomment{French}{Native}{}
\cvitemwithcomment{English}{Fluent}{}

\end{document}

